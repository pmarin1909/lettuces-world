
\chapter{Introduction}
% Hablar de la empresa
% Hablar del objetiivo del trabajo
% 

\chapter{Glosary}
- Topo
- Apero
- Siembra
- Plantación
- Meseta
- Arquillo
- Malla
- Papel
- Farm lanes: caminos entre las mesetas para el paso de la maquinaria.


% Explicación del problema concreto
% variables/constantes
% Funcion objetivo
% 



Buenas tardes a todos, 

Hemos simplificado el problema dejando solo dos fincas porque vamos a trabajar con dos productos distintos: lechugas y espinacas. 
En cada uno de estos productos hemos considerado dos variantes. 
La elección de estos dos productos se debe a las diferencias de cuidados que necesitan pues todas las diferencias en las líneas de 
trabajo se concetran en las variedades escogidas. 
Las tareas que deben considerarse son: 
\begin {enumerate}
    \item Preparación de tierra: se trata de un conjunto de tareas. Se rompe y voltea la tierra para airearla y se abona para dejarla 
    preparada.
    \item Creación de mesetas: con un tractor se generan las mesetas (extensión de tierra ordenada) donde se realizarán las siguientes
    labores con las medidas necesarias para plantar/sembrar (según corresponda)
    \item Colocar papel: se colocan unas láminas de papel para evitar que crezcan malas hierbas alrededor del producto. 
    \item Sembrar/plantar: la siembra es la introducción de semillas en la tierra, mientras que plantar es introducir una planta 
    un poco cultivada previamente por una empresa externa. 
    \item Riego: se deben colocar los aspersores en las laderas de las mesetas. 
    \item Arquillos: se colocan unos arcos metálicos para poder dejar distancia entre el cultivo y la mallo y que se mantenga el 
    calor.
    \item Poner malla: se coloca una malla de poliamida encima de los arquillos para proteger los cultivos y dejamos crecer los cultivos. 
    \item Quitar malla: tras la maduración del cultivo se retira la malla
    \item Quitar arquillos: se retiran los arquillos.
    \item Quitar Riego: se retiran los dispositivos de riego.
    \item Recoger la cosecha: se distingue dos formas de recoger los cultivos. Dependiendo si se necesitan trabajadores que vayan 
    colocando los productos en las cajas o si se hace de forma automática.
    
\end{enumerate}

Como cada tarea necesita una maquinaria distinta, la velocidad a la que se realiza irá variando en función de la misma. 
Los datos proporcionados por la empresa venían dados en km/h, pero como simplificación nosotros hemos traducido estos datos en mesetas/hora.
Otra simplificación necesaria ha sido considerar que nuestras fincas seleccionadas tendrán una medida "media". 
Por otro lado, consideraremos que existen 2 grupos de trabajo independientes, uno para cada producto. 
Así, cada grupo de trabajo tendrá una maquinaria distinta y realizará las tareas de forma independiente, sin tener en consideración como se realicen las tareas en la otra finca.

Estamos considerando que trabajamos en una atmósfera idílica, en la que el clima va a permitir que se puedan realizar todas las labores
de trabajo y no hay ningún tipo de rotura en la maquinaria que genere un retraso en la planificaciión de las tareas.

Tras nuestra visita para ver el funcionamiento de la empresa, vimos como algunos equipos de trabajo se solapaban en una misma 
meseta. Esto generaba horas muertas en algunos grupos de trabajo, ya que no podían empezar hasta que hubiese terminado la anterior. 
Estas hora contabilizan en el computo de horas semanales, pero sin embargo, son horas muertas en las que no se ha realizado ninguna tarea,
solo le cuesta dinero a la empresa. 

El problema trata de minimizar el numero de horas muerta que hay ente los grupos de trabajo. 

Tras un planteamiento más concreto del ejercicio empezamos con la fomulación que tenemos. 

\textbf{Las variables que vamos a considerar son:} 

$X_{ijkl}$ es una variable binaria que nos dice si el equipo de trabajo $i$ esta trabajando en la tierra $j$ en el camino $k$ durante 
la hora $l$.

$W_{il}$ es una variable binaria que nos indica si la tarea $i$ tiene caminos disponibles durante la hora $l$. 

Los índices de las variables $j,k,l$ irán dependiendo de la cantidad de tareas que se tenga que realizar en el cultivo en el que estemos. 
De esta forma reduciremos considerablemente el número de variables y restricciones neccesarias.

Definimos las siguiemtes constantes: 
$T_i$ hace referencia al número máximo de caminos que puede realizar cada tarea $i$. 

$C_i$ son los caminos totales en los que se puede realizar la tarea $i$. 

$L_i$ es la hora estimada a la que debería empezar la tarea $i$. 

\textbf{La funcion objetivo es:}
\[\begin{aligned}
    min \sum_{i,l}\left( \frac{T_i-\sum_{j,k} X_{ijkl}}{T_i}\right)W_{il}
\end{aligned}\]

\textbf{Las restricciones son:}

Cada tarea se realiza una sola vez en cada uno de los caminos 
\[\begin{aligned}
\sum_{l} X_{ijkl} \leq 1 \quad \forall i,j,k
\end{aligned}\]

Una tarea no se puede realizar si no se han realizado las tareas anteriores en ese camino
\[\begin{aligned}
X_{ijkl} \leq X_{(i-1)jkl} \quad \forall i>1,j,k,l
\end{aligned}\]

Número máximo de caminos que realiza cada equipo de trabajo $i$ en la hora $l$
\[\begin{aligned}
\sum_{j.k} X_{ijkl} \leq T_i \quad \forall i,l
\end{aligned}\]

Forzamos que la variable $W_{il}$ sea 1 cuando queden el equipo de trabajo i tenga caminos disponibles en los que trabajar
\[\begin{aligned}
W_{il} \geq \dfrac{C_i-\sum_{j,k} X_{ijkl}}{C_i}  \quad \forall i,l
\end{aligned}\]
