\documentclass[a4paper,11pt,oneside]{report}
\usepackage[table,xcdraw, x11names,dvipsnames,table]{xcolor}

\usepackage{longtable}
\usepackage[utf8]{inputenc}
\usepackage[english]{babel}
\usepackage{courier}
\usepackage{placeins}
\usepackage{amssymb}
\usepackage{amsmath}
\usepackage{graphicx}
\usepackage{tcolorbox}
\usepackage{listings}
\usepackage{graphics}
\usepackage{color}
\usepackage{lipsum}
\usepackage{csquotes}
\usepackage[left=2.00cm, right=2.00cm, top=2.00cm, bottom=2.00cm]{geometry}
\usepackage{array} 
\usepackage{ltablex}
\usepackage{longtable}
\usepackage[strict]{changepage}
\usepackage{tikz}
\usepackage{etoolbox}
\usepackage{multirow}
\usepackage{setspace} 
\usepackage{outlines}
\usepackage{tikz}  %burbujas de texto
\usepackage{textcomp} % si quieres superíndices como "*"
\usepackage[toc,page]{appendix}

% SI NO VAS A USAR GLOSARIO QUITA ESTO
\usepackage{glossaries}    %Glosario
\makeglossaries
\newglossaryentry{raised bed}{
    name=raised bed,
    description={Elevated land surfaces where different crops are cultivated. Their length varies depending on the size of the farms, but they are usually 1.6 meters wide.}
}
\newglossaryentry{slope}{
    name=slope,
    description={Narrow strips of land between the plateaus that allow farmers and tractors to work without damaging the crops.}
	}
\newglossaryentry{transplanted}{
    name=transplanting,
    description={The introduction of plants in their initial stage of development into the soil.
	}
}
\newglossaryentry{Sowing}{
    name=sowing,
    description={Scattering seeds directly onto prepared soil for cultivation.}
}
\newglossaryentry{mesh}{
    name=mesh,
    description={Flexible structures placed over crops to protect the plants from external factors.
	}
}
\newglossaryentry{hoops}{
    name=hoops,
    description={Usually iron arches placed on the plateaus. The mesh is often placed on top, creating a space between the crops and the mesh.
	}
}
\newglossaryentry{harvest}{
    name=harvesting,
    description={Collecting the different products once they have matured.}
}




%\item Plateau: Elevated land surfaces where different crops are cultivated. Their length varies depending on the size of the farms, but they are usually 1.6 meters wide.
		
%\item Slope: Narrow strips of land between the plateaus that allow farmers and tractors to work without damaging the crops.

%\item Planting: The introduction of plants in their initial stage of development into the soil.

%\item Sowing: Scattering seeds directly onto prepared soil for cultivation.

%\item Mesh: Flexible structures placed over crops to protect the plants from external factors.

%\item Hoops or arches: Usually iron arches placed on the plateaus. The mesh is often placed on top, creating a space between the crops and the mesh.

%\item Harvesting: Collecting the different products once they have matured. % definiciones externas

% -----------------------------------------




\usepackage{hyperref}
\hypersetup{
    colorlinks=true,
    linkcolor=grey3,
    filecolor=magenta,      
    urlcolor=azul5,
    pdfpagemode=FullScreen,
    }
\urlstyle{same}



\newcommand{\burbuja}[2]{%
    \tikz[baseline=(bubble.center)]{
        \node[draw, fill=blue!20, text width=#1, align=center, rounded corners] (bubble) {#2};
    }%
}

\DeclareUnicodeCharacter{2212}{-}

% SI NO VAS A USAR BIBLIOGRAFIA QUITA ESTO
\usepackage[backend=biber]{biblatex}
\addbibresource{dummy.bib}
% -----------------------------------------

\usepackage{booktabs}

%Los colores están enumerados de menos a mas tono y saturación
% PALETA DE COLORES: https://www.rapidtables.com/web/color/RGB_Color.html
\definecolor{azul1}{RGB}{204, 229, 255}
\definecolor{azul2}{RGB}{153, 204, 255}
\definecolor{azul3}{RGB}{102, 178, 255}
\definecolor{azul4}{RGB}{51, 153, 255}
\definecolor{azul5}{RGB}{0, 128, 255}
\definecolor{azul6}{RGB}{0, 102, 204}
\definecolor{azul7}{RGB}{0, 76, 153}
\definecolor{azul8}{RGB}{0, 51, 102}
\definecolor{azul_lila1}{RGB}{204, 204, 255}
\definecolor{azul_lila2}{RGB}{153, 153, 255}    
\definecolor{azul_lila3}{RGB}{102, 102, 255}
\definecolor{azul_lila4}{RGB}{51, 51, 255}
\definecolor{azul_lila5}{RGB}{0, 0, 255}
\definecolor{azul_lila6}{RGB}{0, 0, 204}
\definecolor{azul_lila7}{RGB}{0, 0, 153}
\definecolor{azul_lila8}{RGB}{0, 0, 102}

\definecolor{purple_ubuntu}{RGB}{0.172, 0.000, 0.118}
\definecolor{purple_blue}{RGB}{142,170,219}
\definecolor{rojo}{RGB}{224, 59, 35}
\definecolor{grey1}{RGB}{238, 237, 237}
\definecolor{grey2}{RGB}{180, 178, 177}
\definecolor{grey3}{RGB}{67,75, 77}

%Colores terminal
\definecolor{terminal-bg}{rgb}{0.2, 0.2, 0.2}  % Fondo oscuro
\definecolor{terminal-fg}{rgb}{1.0, 1.0, 1.0}  % Texto blanco
\definecolor{terminal-green}{rgb}{0.5, 1.0, 0.5} % Verde para el usuario
\definecolor{terminal-blue}{rgb}{0.4, 0.5, 1.0} % Azul para el directorio

%Caja de la terminal
\newtcolorbox{terminalbox}{colback=terminal-bg, colframe=terminal-bg, 
  boxrule=0pt, left=3pt, right=3pt, top=3pt, bottom=3pt, 
  sharp corners, boxsep=5pt, width=\linewidth}


% Definir comando para formatear automáticamente el prompt
\newcommand{\terminalPrompt}[4]{%
   \textbf{\textcolor{terminal-green}{#1}} % Usuario en verde
   \textbf{\textcolor{terminal-fg}{@}}     % Separador en blanco
    \textbf{\textcolor{terminal-blue}{#2}}  % Host en azul
    \textbf{\textcolor{terminal-fg}{:}}
    \textbf{\textcolor{terminal-blue}{$\thicksim$}}
    \textbf{\textcolor{terminal-blue}{#3}}
    \textbf{\textcolor{terminal-fg}{\$}}   % Separador en blanco
    \textcolor{terminal-fg}{#4}    % Resto del texto en blanco
}


\tcbuselibrary{listings,breakable}
\setlength{\parskip}{3mm}
\tcbset{colback=red!5!white,colframe=red!75!black,size=small,
fonttitle=\bfseries,box align=top}
\lstdefinestyle{customc}{
  breaklines=true,
  language=python,
  basicstyle=\footnotesize\ttfamily,
  keywordstyle=\bfseries\color{purple!40!black},
  commentstyle=\itshape\color{green!40!black},
  identifierstyle=\color{blue},
  stringstyle=\color{orange},
}



\begin{document}
    \begin{titlepage}
    \begin{center}
        {\Huge \textsc{Harvesting Efficiency:}}

        {\Huge \textsc{Mathematics applied to the land}}  
        
        
        {\Large{\textbf{Modeling Laboratory - Group Work. Optimization}}}
        
        \vspace{20mm}
        {\large{ 
          Juan Agustín Lorca García

          Paula Marín Turpín
          
          Andrea Martos García

          Rebeca Molina Bernal
          
          Date: 14 March 2025 
          }}

          \vspace{20mm}
        \begin{figure}[ht]
             \centerline{\includegraphics[width=10cm,height=10cm]{portada/nuevo_logo_umu.png}}
            \label{logo_umu}
        \end{figure}

        \vspace{20mm}
        {\large {University of Murcia}}
        
        {\large {Faculty of Mathematics}}
        \textcolor{rojo}{\rule{\linewidth}{0.55mm}}
    \end{center}
\end{titlepage}
    \include{portada/indice.tex}
    \chapter*{Introduction}
\addcontentsline{toc}{chapter}{1. Introduction} % Manual index entry
% Hablar de la empresa
% Hablar del objetivo del trabajo
% Comentar la estructura del documento

En el mundo actual, donde la eficiencia y la toma de decisiones estratégicas marcan la diferencia entre avanzar o quedarse atrás, la gestión de tareas dentro de una empresa adquiere un papel fundamental.
Muchas veces, detrás de un buen producto o servicio, hay una planificación precisa, una coordinación cuidada y un aprovechamiento óptimo de los recursos disponibles.
En este contexto, las matemáticas, y más concretamente la optimización, se convierten en una herramienta poderosa para transformar la complejidad en soluciones claras y aplicables.

Este trabajo surge con la intención de tender un puente entre el conocimiento teórico y su aplicación práctica en el entorno empresarial.
Hemos elegido como caso de estudio la empresa Intercrop, ubicada en Cartagena, especializada en el sector agroalimentario.
Intercrop destaca por su compromiso con la sostenibilidad y la innovación en la producción agrícola, pero como toda empresa, se enfrenta a retos logísticos y organizativos que requieren soluciones inteligentes.
Intercrop no solo opera a nivel nacional, sino que también mantiene una estrecha relación con el mercado internacional. 
Exporta una parte importante de su producción a distintos países europeos, lo que exige altos estándares de calidad, cumplimiento riguroso de plazos y una logística bien estructurada.
Esta dimensión internacional añade complejidad a su gestión operativa, ya que debe coordinar las tareas agrícolas con los calendarios de transporte, las exigencias fitosanitarias y los compromisos comerciales en el extranjero.
Todo ello convierte a esta empresa en un entorno especialmente interesante para aplicar herramientas de optimización que ayuden a mejorar la planificación y la eficiencia en un contexto real y exigente.

A través de este proyecto, abordaremos el problema de planificación de tareas dentro de la empresa. La meta es diseñar un modelo de optimización que permita organizar de forma eficiente las actividades,
 considerando las restricciones del entorno real: tiempos, recursos limitados, dependencias entre tareas y otros factores logísticos.
Este proceso nos permitirá no solo aportar una propuesta de mejora a la empresa, sino también aplicar de forma práctica los conceptos matemáticos aprendidos en el aula, especialmente en lo que respecta a programación lineal y optimización.

\chapter*{Problem Statement}
\addcontentsline{toc}{chapter}{2. Problem Statement} % Manual index entry
% Explicación en "leguaje natural" del problema
% Simplificaciones
La gestión eficiente de las tareas agrícolas representa un reto significativo en el día a día de una empresa como Intercrop, especialmente cuando se trabaja con distintos productos, cada uno con requerimientos técnicos y logísticos específicos.
En este proyecto, hemos decidido abordar una versión simplificada y controlada del problema real, manteniendo, sin embargo, la complejidad suficiente como para reflejar los principales desafíos que se presentan en la planificación de tareas en un entorno agrícola profesional.

Con este objetivo, hemos centrado nuestro estudio en dos fincas concretas, cada una dedicada al cultivo de un producto diferente: lechuga y espinaca.
La elección de estos dos productos no ha sido aleatoria; se debe a las marcadas diferencias en sus necesidades de cuidado y manejo.
De hecho, es en estas diferencias donde se concentran gran parte de las particularidades de las líneas de trabajo dentro de la empresa.
Para enriquecer aún más el modelo y hacerlo más representativo, hemos considerado dos variantes dentro de cada uno de estos productos, lo que nos permite capturar con mayor fidelidad la diversidad de cultivos con la que trabaja Intercrop en la práctica.

En cada finca se lleva a cabo una secuencia de tareas específicas necesarias para completar el ciclo productivo de forma adecuada. Estas tareas, que constituyen el núcleo del proceso agrícola, son las siguientes:
\begin{enumerate}
    \item Preparación de la tierra: conjunto de labores inicales donde se rompe y voltea la tierra para airearla, y se abona para dejarla preparada adecuadamente antes del cultivo.
    \item Creación de mesetas: mediante maquinaria especializada, se organizan las superficies de cultivo en mesetas 
\end{enumerate}


El \gls{tempo} de una pieza puede variar según el estilo.



\chapter*{Formulation}
\addcontentsline{toc}{chapter}{3. Formulation} % Manual index entry

% Variables
% Funcion objetivo
% Restricciones

\chapter*{Data}
\addcontentsline{toc}{chapter}{4. Data} % Manual index entry
% Explicación de los datos
% Tablas

Durante nuestra visita a la empresa, recopilamos todos los datos necesarios para nuestro análisis. 
Posteriormente, la empresa nos facilitó la información que habíamos solicitado.

Para nuestro estudio, asumimos que las fincas tienen una forma rectangular, con dimensiones promedio
 de 100 x 300 metros. Cada meseta de cultivo mide 1.6 metros de ancho, con surcos laterales de 0.4 metros
  para permitir el paso de la maquinaria. Esto implica que cada camino tendrá unas dimensiones de 2 x 300 metros.
   En consecuencia, en cada finca disponemos de 50 caminos y sus respectivas 50 mesetas para el cultivo.

Dado que todas las tareas están mecanizadas, es fundamental considerar la velocidad de las máquinas utilizadas en el
 proceso. Aunque la empresa nos proporcionó estos datos en kilómetros por hora (km/h), para facilitar nuestra formulación 
 convertimos las unidades a caminos por hora.
 
 \begin{table}[ht!]
    \begin{tabular}{lclll}
    \textbf{TASK}      & \multicolumn{1}{l}{\textbf{SPEED}} \\
    Soil preparation   & 2                                  \\
    Install paper      & 4                                  \\
    Plant              & 3                                  \\
    Sow                & 4                                  \\
    Install irrigation & 3                                  \\
    Install hoops      & 3                                  \\
    Install mesh       & 3                                  \\
    Remove irrigation  & 3                                  \\
    Remove mesh        & 4                                  \\
    Remove hoops       & 3                                  \\
    Harvest seeds      & 3                                  \\
    Harvest plants     & 2                                            
    \end{tabular}
    \end{table}

Nuestra planificación está diseñada para organizar un mes de trabajo. Considerando una jornada laboral de 8 horas diarias, 
trabajamos con un total de 240 horas al mes.

Por otro lado, los trabajadores se incorporan a la campaña de manera escalonada. Cada uno de ellos está especializado en una 
tarea específica, lo que da lugar a la formación de grupos de trabajo especializados. El número de trabajadores por grupo varía 
en función de la tarea asignada.



\chapter*{Code}
\addcontentsline{toc}{chapter}{5. Code} % Manual index entry\addcontentsline{toc}{chapter}{5. Code} % Manual index entry
% Explicación del código
% Código

\chapter*{Results}
\addcontentsline{toc}{chapter}{6. Results} % Manual index entry
% Explicación de los resultados
% Tablas
% Gráficos

\chapter*{Conclusion}
\addcontentsline{toc}{chapter}{7. Conclusion} % Manual index entry

\newpage
\appendix
\chapter*{Glosary}
\addcontentsline{toc}{chapter}{A. Glosary} % Manual index entry


- Topo
- Apero
- Siembra
- Plantación
- Meseta
- Arquillo
- Malla
- Papel
- Farm lanes: caminos entre las mesetas para el paso de la maquinaria.





Buenas tardes a todos, 

Hemos simplificado el problema dejando solo dos fincas porque vamos a trabajar con dos productos distintos: lechugas y espinacas. 
En cada uno de estos productos hemos considerado dos variantes. 
La elección de estos dos productos se debe a las diferencias de cuidados que necesitan pues todas las diferencias en las líneas de 
trabajo se concetran en las variedades escogidas. 
Las tareas que deben considerarse son: 
\begin {enumerate}
    \item Preparación de tierra: se trata de un conjunto de tareas. Se rompe y voltea la tierra para airearla y se abona para dejarla 
    preparada.
    \item Creación de mesetas: con un tractor se generan las mesetas (extensión de tierra ordenada) donde se realizarán las siguientes
    labores con las medidas necesarias para plantar/sembrar (según corresponda)
    \item Colocar papel: se colocan unas láminas de papel para evitar que crezcan malas hierbas alrededor del producto. 
    \item Sembrar/plantar: la siembra es la introducción de semillas en la tierra, mientras que plantar es introducir una planta 
    un poco cultivada previamente por una empresa externa. 
    \item Riego: se deben colocar los aspersores en las laderas de las mesetas. 
    \item Arquillos: se colocan unos arcos metálicos para poder dejar distancia entre el cultivo y la mallo y que se mantenga el 
    calor.
    \item Poner malla: se coloca una malla de poliamida encima de los arquillos para proteger los cultivos y dejamos crecer los cultivos. 
    \item Quitar malla: tras la maduración del cultivo se retira la malla
    \item Quitar arquillos: se retiran los arquillos.
    \item Quitar Riego: se retiran los dispositivos de riego.
    \item Recoger la cosecha: se distingue dos formas de recoger los cultivos. Dependiendo si se necesitan trabajadores que vayan 
    colocando los productos en las cajas o si se hace de forma automática.
    
\end{enumerate}

Como cada tarea necesita una maquinaria distinta, la velocidad a la que se realiza irá variando en función de la misma. 
Los datos proporcionados por la empresa venían dados en km/h, pero como simplificación nosotros hemos traducido estos datos en mesetas/hora.
Otra simplificación necesaria ha sido considerar que nuestras fincas seleccionadas tendrán una medida "media". 
Por otro lado, consideraremos que existen 2 grupos de trabajo independientes, uno para cada producto. 
Así, cada grupo de trabajo tendrá una maquinaria distinta y realizará las tareas de forma independiente, sin tener en consideración como se realicen las tareas en la otra finca.

Estamos considerando que trabajamos en una atmósfera idílica, en la que el clima va a permitir que se puedan realizar todas las labores
de trabajo y no hay ningún tipo de rotura en la maquinaria que genere un retraso en la planificaciión de las tareas.

Tras nuestra visita para ver el funcionamiento de la empresa, vimos como algunos equipos de trabajo se solapaban en una misma 
meseta. Esto generaba horas muertas en algunos grupos de trabajo, ya que no podían empezar hasta que hubiese terminado la anterior. 
Estas hora contabilizan en el computo de horas semanales, pero sin embargo, son horas muertas en las que no se ha realizado ninguna tarea,
solo le cuesta dinero a la empresa. 

El problema trata de minimizar el numero de horas muerta que hay ente los grupos de trabajo. 

Tras un planteamiento más concreto del ejercicio empezamos con la fomulación que tenemos. 

\textbf{Las variables que vamos a considerar son:} 

$X_{ijkl}$ es una variable binaria que nos dice si el equipo de trabajo $i$ esta trabajando en la tierra $j$ en el camino $k$ durante 
la hora $l$.

$W_{il}$ es una variable binaria que nos indica si la tarea $i$ tiene caminos disponibles durante la hora $l$. 

Los índices de las variables $j,k,l$ irán dependiendo de la cantidad de tareas que se tenga que realizar en el cultivo en el que estemos. 
De esta forma reduciremos considerablemente el número de variables y restricciones neccesarias.

Definimos las siguientes constantes: 
$T_i$ hace referencia al número máximo de caminos que puede realizar cada tarea $i$. 

$C_i$ son los caminos totales en los que se puede realizar la tarea $i$. 

$L_i$ es la hora estimada a la que debería empezar la tarea $i$. 

\textbf{La funcion objetivo es:}
\[\begin{aligned}
    min \sum_{i,l}\left( \frac{T_i-\sum_{j,k} X_{ijkl}}{T_i}\right)W_{il}
\end{aligned}\]

\textbf{Las restricciones son:}

Cada tarea se realiza una sola vez en cada uno de los caminos 
\[\begin{aligned}
\sum_{l} X_{ijkl} \leq 1 \quad \forall i,j,k
\end{aligned}\]

Una tarea no se puede realizar si no se han realizado las tareas anteriores en ese camino
\[\begin{aligned}
X_{ijkl} \leq X_{(i-1)jkl} \quad \forall i>1,j,k,l
\end{aligned}\]

Número máximo de caminos que realiza cada equipo de trabajo $i$ en la hora $l$
\[\begin{aligned}
\sum_{j.k} X_{ijkl} \leq T_i \quad \forall i,l
\end{aligned}\]

Forzamos que la variable $W_{il}$ sea 1 cuando queden el equipo de trabajo i tenga caminos disponibles en los que trabajar
\[\begin{aligned}
W_{il} \geq \dfrac{C_i-\sum_{j,k} X_{ijkl}}{C_i}  \quad \forall i,l
\end{aligned}\]

    
    \printglossaries
    \addcontentsline{toc}{chapter}{A. Glossary} % Manual index entry

    %\printbibliography
\end{document}
