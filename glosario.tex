\newglossaryentry{raised bed}{
    name=raised bed,
    description={Elevated land surfaces where different crops are cultivated. Their length varies depending on the size of the farms, but they are usually 1.6 meters wide.}
}
\newglossaryentry{slope}{
    name=slope,
    description={Narrow strips of land between the plateaus that allow farmers and tractors to work without damaging the crops.}
	}
\newglossaryentry{transplanted}{
    name=transplanting,
    description={The introduction of plants in their initial stage of development into the soil.
	}
}
\newglossaryentry{Sowing}{
    name=sowing,
    description={Scattering seeds directly onto prepared soil for cultivation.}
}
\newglossaryentry{mesh}{
    name=mesh,
    description={Flexible structures placed over crops to protect the plants from external factors.
	}
}
\newglossaryentry{hoops}{
    name=hoops,
    description={Usually iron arches placed on the plateaus. The mesh is often placed on top, creating a space between the crops and the mesh.
	}
}
\newglossaryentry{harvest}{
    name=harvesting,
    description={Collecting the different products once they have matured.}
}




%\item Plateau: Elevated land surfaces where different crops are cultivated. Their length varies depending on the size of the farms, but they are usually 1.6 meters wide.
		
%\item Slope: Narrow strips of land between the plateaus that allow farmers and tractors to work without damaging the crops.

%\item Planting: The introduction of plants in their initial stage of development into the soil.

%\item Sowing: Scattering seeds directly onto prepared soil for cultivation.

%\item Mesh: Flexible structures placed over crops to protect the plants from external factors.

%\item Hoops or arches: Usually iron arches placed on the plateaus. The mesh is often placed on top, creating a space between the crops and the mesh.

%\item Harvesting: Collecting the different products once they have matured.